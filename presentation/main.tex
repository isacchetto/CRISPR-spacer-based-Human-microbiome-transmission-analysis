% arara: pdflatex: { shell: yes, synctex: yes, options: [--output-directory=build] }
% arara: move: { files: [ 'build/main.pdf' ], target: 'backup/main.pdf' }
% arara: clean: { extensions: [ 'aux', 'log', 'toc', 'out', 'blg', 'bbl', 'bcf', 'glo', 'gls' ] }

\documentclass{beamer}

\usepackage{graphicx} % Per includere immagini

\renewcommand{\familydefault}{\sfdefault}

\title{Tirocinio}
\author{Isacco Cenacchi}
\date{\today}

\begin{document}

\frame{\titlepage}

% Slide 1
\begin{frame}
    \frametitle{\ttfamily CRISPR totali vs CRISPR con cas operon vicino}
    \centering
    \includegraphics[width=\linewidth]{../figs/CRISPR_totali_vs_CRISPR_cas_operon_vicino.png}
    Vediamo che i CRISPRs associati a Cas Operon non aumentano troppo significativamente usando parametri piu rilassati
\end{frame}


\begin{frame}
    \frametitle{\ttfamily Numero totale di spacers distanti <10000bp da cas operon}
    \begin{minipage}{0.7\textwidth}
        \centering
        \includegraphics[width=\textwidth, height=0.9\textheight, keepaspectratio]{../figs/Numero_totale_spacers_cas_operon.png}
    \end{minipage}%
    \hfill
    \begin{minipage}{0.25\textwidth}
        Anche dal numero di spacers (vicino ai cas operon) trovati da ogni tool vediamo che non aumentano di molto con i parametri più rilassati.
    \end{minipage}
\end{frame}

\begin{frame}
    \frametitle{\ttfamily Unique DR all'aumentare dei tool}
    \centering
    \includegraphics[width=\linewidth]{../figs/Unique_DR_cumulative_and_individual.png}
\end{frame}

\begin{frame}
    \frametitle{\ttfamily Upset plot CRISPR}
    \centering
    \includegraphics[width=\linewidth]{../figs/Upset_plot_CRISPR.png}
    Osserviamo che i CRISPRs trovati da ogni tool sono molto diversi tra loro. e ne vengono trovati pochi in comune.
\end{frame}

\begin{frame}
    \frametitle{\ttfamily Upset plot ClusterID CRISPR}
    \centering
    \includegraphics[width=\linewidth]{../figs/Upset_plot_ClusterID.png}
    Osserviamo che invece raggruppando i CRISPRs in ClusterID, osserviamo che i CRISPRs trovati dalla maggior parte dei tool sono quelli di "alta qualità", mentre i CRISPRs che non sono anche quelli piu lontani da cas operon.
\end{frame}

\end{document}
